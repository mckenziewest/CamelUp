\documentclass{article}
\usepackage[utf8]{inputenc}
\usepackage{amsmath,amsthm,amssymb,graphicx,hyperref}
\usepackage[margin=1in]{geometry}


\newcommand{\N}{\mathbb{N}}
\newcommand{\Q}{\mathbb{Q}}
\newcommand{\R}{\mathbb{R}}
\newcommand{\Z}{\mathbb{Z}}

\newcommand{\abar}{\overline{a}}
\newcommand{\bbar}{\overline{b}}
\date{October 2021}

\begin{document}
\title{Theorem Proof}
\maketitle
\textbf{Definition 1: Camel Unit}\\ 
A camel unit is a collection of camels that is stacked on top of each other. Movement is defined such that whenever a camel moves, all camels that are stacked on top of it in the unit move as well and maintain the same position relative to the other camels in the new unit.
\textbf{Statement:} If a camel unit is of length 2 or greater at the start of a leg, any camel on that unit of height 2 or higher is guaranteed to move, provided that neither of the bottom two are the white nor the black camel. \\
\begin{proof}
Suppose it is the start of a leg. Because of how movement is defined, it will suffice to show that the camel at height 3 is guaranteed to move, i.e. that one of the bottom 3 camels is guaranteed to move, so suppose we have a camel unit of height 3. There are 3 separate cases for this. \\
\textbf{Case 1:} Both of the bottom camels move. \\
If both of the bottom camels move, then all camels in the unit will move by virtue of how movement is defined and the fact that the very bottom camel moves. \\
\textbf{Case 2:} The very bottom camel moves. \\
If the very bottom camel moves, then as above, all camels in the movement will also move. \\
\textbf{Case 3:} The second camel from the bottom moves. \\
If the second camel from the bottom moves, then all camels above it will also move by definition of how movement works. \\
Because there are 6 dice and we are excluding the black and white camel, which share a unique die, from consideration in this scenario, we thus find that all the camels of height 2 or greater move. \\
\end{proof}

\end{document}
