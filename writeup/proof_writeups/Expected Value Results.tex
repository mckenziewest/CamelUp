\documentclass{article}
\usepackage{amsmath,amsfonts,amsthm,amssymb,mathrsfs,graphicx,adjustbox,enumitem}
\usepackage[utf8]{inputenc}
\newtheorem{theorem}{Theorem}
\title{A Simpler Method for Finding Expected Value in Camel Up}
\author{Kyle Eckland}
\date{November 2021}

\begin{document}
\maketitle

\begin{theorem}[Expected Value for Stacked Camels]
Suppose it is the start of a new leg and we have a stack of 2 or more racing camels with no camels in front of them. Let $X$ be the discrete random variable representing the distance a given camel from that stack can move in this leg, and assume $Z$ is the random variable that represents the number of times that camel can move in this leg. Then, $E[X] = 2E[Z]$
\end{theorem}

\begin{proof}
 Generally, the expected value of a discrete random variable $X$ is defined as $E[X] = \sum_{k = 1}^{r}kp(X=k)$, where $p(X = k)$ is the probability that $X = k$. In the context of movement in camel up, $r$ would be the furthest distance a camel could move. If we were to try to apply this method to the expected value of a camel moving $k$ spaces in a given leg, we would have to first have to find each way the camel could move a distance of $k$, which can be difficult.  Instead we will switch our perspective: rather than looking at ways a camel can move a given distance, we will look at the number of ways a camel moves a given number of times within a leg. We'll denote this with the random variable $Z$. Then, the expected number of times a camel moves is $\sum_{z=1}^{n}zp(z)$, where $n$ is the maximum number of times a camel can move in a turn. Notice that each time a camel moves requires a dice roll, and we know that the expected value of a single dice roll here is $2$. Since the value of an individual roll is independent from the number of dice rolled or the order they're rolled in, if we multiply the expected value of a dice roll by the expected value of the number of times a camel moves in a turn, we find the expected distance a camel moves in a turn to be $E[X] = 2E[Z]$.
\end{proof}
\end{document}
