\documentclass{article}
\usepackage[utf8]{inputenc}

\title{A Simpler Method for Finding Expected Value in Camel Up}
\author{Kyle Eckland}
\date{November 2021}

\begin{document}
\maketitle

\section{A New Formulation}
\\ Generally, the expected value of a discrete random variable $X$ is defined as $E[X] = \sum_{k = 1}^{r}kp(X=k)$, where $p(X = k)$ is the probability that $X = k$. In the context of movement in camel up, if we let $X$ be the random variable representing the distance a camel moved in a given leg, then $r$ would be the furthest distance a camel could move. If we were to try to apply this method to the expected value of a camel moving $k$ spaces in a given leg, we would have to first have to find each way the camel could move a distance of $k$, which can be difficult. This method is tedious and prone to errors, so instead we'll use a method that paints the same picture with more efficient brush strokes. To do this, we will switch our perspective: rather than looking at ways a camel can move a given distance, we will look at the number of ways a camel moves a given number of times within a leg. We'll denote the random variable that represents the number of times a camel moves in a given turn as $Z$. Then, the expected number of times a camel move is $\sum_{z=1}^{n}zp(z)$, where $n$ is the total number of times a camel can move in a turn. Notice that each time a camel moves requires a dice roll, and we know that the expected value of a single dice roll here is $2$. Since the value of an individual roll is independent from the number of dice rolled or the order they're rolled in, if we multiply the expected value of a dice roll by the expected value of the number of times a camel moves in a turn, we find the expected distance a camel moves in a turn. Thus, $E[X] = 2E[Z]$. This allows for a simpler way to find the expected value by means of calculating the probability of the number of possible movements.

\end{document}
