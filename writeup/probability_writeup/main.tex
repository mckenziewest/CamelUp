\documentclass{article}
\usepackage[utf8]{inputenc}

\title{Probability Info Write-Up}
\author{kyjeckland }
\date{October 2021}

\begin{document}

\maketitle

\section{Desert Tiles}
Here, we look at the probabilities and expected values associated with desert tiles in the simplest case of only one camel. In this instance, the camel can move at most 3 spaces, so we will restrict the distance of the desert tile to within 3 tiles. Notice that each possible movement value is equiprobable, so the probability that the camel lands on the desert tile is exactly $\frac{1}{3}$ regardless of the spot selected for the desert tile. Next, we will take a look at the expected value in the three different cases.

\textbf{Case 1:} The desert tile is immediately in front of the camel.In this case, if a $1$ is rolled the camel will move $2$ spaces. If we let $Z$ be the random value that represents the distance the camel moves, then the expected value is calculated as follows:

\begin{center} $E(Z) = \frac{1}{3}*2 + \frac{1}{3}*2+\frac{1}{3}*3 = \frac{2}{3}+\frac{2}{3}+\frac{3}{3} = \frac{7}{3} \approx 2.33$
\end{center}

\textbf{Case 2:} The desert tile is two tiles in front of the camel. If a $2$ is rolled here, then the camel will move $3$ spaces. As above, we find the expected value as follows:

\begin{center} $E(Z) = \frac{1}{3}*1+\frac{1}{3}*3+\frac{1}{3}*3 = \frac{7}{3} \approx 2.33$
\end{center}

\textbf{Case 3:} The third and final case is when the desert tile is 3 spaces away. Here, if 3 is rolled, the camel will move 4 spaces. The expected value is then:
\begin{center} $E(Z) = \frac{1}{3}*1+\frac{1}{3}*2+\frac{1}{3}*4 = \frac{7}{3} \approx 2.33$
\end{center}

\section{Crazy Camels}
The crazy camels move in the opposite direction of the other camels, but only pose a threat if a camel lands on top of them. Here we will go over the simplest case of one normal camel and one crazy camel, and evaluate the probabilities and expected values in each case. The probabilities in this scenario are more complex than those in the desert tile scenario. We will assume that both the crazy camel and normal camel's dice are the only ones that have yet to be rolled. Notice that if the crazy camel goes first, the probability that the normal camel moves backward is exactly 0. Furthermore, the probability of this specific crazy camel moving upon the crazy die being rolled is $.5$, because there are $3$ ways it can move out of a total of $6$ possibilities. Suppose $x$ is the distance of the crazy camel from the normal camel. Given this information, it follows that the probability the normal camel will move backward is 
 $&=\frac{1}{3}\frac{1}{2}\frac{1}{2} = \frac{1}{12}$

The probability that the normal camel lands on the crazy camel but the crazy camel doesn't move would then be 
\begin{center}$\frac{1}{2}\frac{1}{3}\frac{1}{2}+\frac{1}{2}\frac{1}{3} = \frac{1}{12}+\frac{1}{6} = \frac{1}{12}+\frac{2}{12} = \frac{3}{12} = \frac{1}{4}$ 
\end{center}
With this information, we can then calculate the expected values of each possible distance.

\textbf{1 tile away:}
\\\begin{center} $E(Z) = 1*\frac{1}{4} + 2*\frac{1}{3}+3*\frac{1}{3} -\frac{1}{36}+\frac{2}{36} = \frac{11}{6} \approx 1.83$
\end{center}

\textbf{2 tiles away:}
\\\begin{center} $E(Z) = 1*\frac{1}{3} + 2*\frac{1}{4}+3*\frac{1}{3} -\frac{1}{36}+\frac{1}{36} = \frac{11}{6} \approx 1.83$
\end{center}

\textbf{3 tiles away:}
\\\begin{center} $E(Z) = 1*\frac{1}{3} + 2*\frac{1}{3}+3*\frac{1}{4} +\frac{2}{36}+\frac{1}{36} = \frac{11}{6} \approx 1.83$
\end{center}

And so the expected values are all the same, as with the desert tiles. How, then, do we judge which position is 'better' or 'worse,' and can we generalize this sort of thinking to apply to more complex scenarios?





\end{document}
