\documentclass{article}
\usepackage[utf8]{inputenc}

\title{A Simpler Method for Finding Expected Value in Camel Up}
\author{kyjeckland }
\date{November 2021}

\begin{document}
\maketitle

\section{A New Formulation}
\\ Generally, the expected value of a discrete random variable $X$ is defined as $E[X] = \sum_{k = 1}^{r}kp(X=k)$, where $p(X = k)$ is the probability that $X = k$, and $r$ is the furthest distance a camel could move on that turn. If we were to try to apply this method to the expected value of a camel moving $k$ spaces in a given leg, we would have to first have to find each way the camel could move a distance of $k$, which can be difficult. This method is tedious and prone to errors, so instead we'll use a method that paints the same picture with more efficient brush strokes. To do this, we will switch our perspective: rather than looking at ways a camel can move a given distance, we will look at the number of ways a camel moves a given number of times within a leg. We'll denote the random variable that represents the number of times a camel moves in a given turn as $Z$. Then, the expected number of times a camel move is $\sum_{z=1}^{n}zp(z)$, where $n$ is the total number of times a camel can move in a turn. Notice that each time a camel moves requires a dice roll, and we know that the expected value of a single dice roll here is $2$. Since the value of an individual roll is independent from the number of dice rolled or the order they're rolled in, if we multiply the expected value of a dice roll by the expected value of the number of times a camel moves in a turn, we find the expected distance a camel moves in a turn. Thus, $E[X] = 2E[Z]$. This allows for a simpler way to find the expected value by means of calculating the probability of each individual distance.

\section{Analysis}
\\With our new tool, let's put it to the test and get some results. First, let's look at the top camel in a stack of three.

\\\textbf{Stack of 3}
\\It is clear that this camel can move at most $3$ times. The total number of permutations of the dice rolls here is $3! = 6$. We find that there is one case where the camel moves once, and here the camel's die must be rolled before the other two, so $P(Z=1) = \frac{1}{3}$. There are two cases where the camel moves twice. The first, where the middle camel moves first and then the top camel moves, and since the yellow's movement afterwords doesn't affect this scenario it has probability $\frac{1}{3}$. The second has the bottom camel moving and then the top camel moving immediately after, with probability $\frac{1}{6}$, so $P(Z=2) = \frac{1}{3} + \frac{1}{6} = \frac{1}{2}$. Finally, there is only one way in which the top camel moves three times, so $P(Z = 3) = \frac{1}{6}$. Thus, we find that the expected number of times the top camel moves is $E[Z] = \sum{z=1}^{3}zp(z) =1*\frac{1}{3} + 2*\frac{1}{2} + 3*\frac{1}{6} = \frac{11}{6}$. We can then find that the expected value of the top camel's movement that turn is $E[X]=2*\frac{11}{6} = \frac{11}{3} \approx 3.67$.

\\\textbf{Stack of 4} 
\\Now for a more interesting example, finding the expected value of the top camel in a stack of $4$. As before, we will begin by finding the number of ways the camel moves $1$ to $4$ times. The total permutations of dice rolls here is $4! = 24$. There is one way the top camel moves once, and it occurs with probability $1/4$. Next we find the number of ways the top camel moves twice, which are broken down into $3$ types of scenarios. 
The first is when the bottom camel moves and the top camel follows immediately after, which has $2$ different ways of occurring. The second way is when the second to bottom camel moves first, in which case the bottom camel's order of movement doesn't matter but the second to top's does. This type of scenario has $3$ possibilities. The third scenario type is when the second to top camel moves and then the top camel moves. In this case the very bottom camel and the second to bottom camel's movement order doesn't matter, and so we find that there are $6$ ways this scenario occurs. Thus, there are $2 + 3 6 = 11$ different independent ways the top camel moves twice, and so the probability the top camel moves twice is $\frac{11}{24}$ There are $2$ different scenario types in which the top camel moves $3$ times. The first is when the bottom camel moves first, and there are $3$ different variations here where the top camel moves $3$ times. The second type of scenario is when the second to bottom camel moves, and similarly there are $3$ variations here where the top camel moves $3$ times, and so there are $3 + 3 = 6$ ways the top camel moves $3$ times, and so it has probability $\frac{6}{24} = \frac{1/4}$. Finally, there is one way the top camel moves $4$ times, so it has probability $\frac{1}{24}$. Finally, the expectation of the movement of the top camel here is: 
$$E[X]=\sum_{k=1}^4 kp(k) = 2(1*\frac{1}{4} + 2*\frac{11}{24} + 3*\frac{1}{4} + 4*\frac{1}{24}) = 2(\frac{1}{4}+\frac{11}{12}+\frac{3}{4}+\frac{1}{6}) = \frac{25}{6} \approx 4.167$$

\\\textbf{Stack of 5}
\\ Although unlikely, the expected value of a stack of five can be found with this method as well. We find that $P(Z=1) = \frac{1}{5}$, $P(Z=2) = \frac{49}{120}$, $P(Z=3) = \frac{3}{10}$, $P(Z=4) = \frac{1}{12}$, and $P(Z=5+ = \frac{1}{120}$. Thus, we find that $E(Z) = \frac{55}{24}$, and so $E(X) = 2E(Z) = 2*\frac{55}{12} \approx 4.5833$
\end{document}
